%
% CSE Electronic Homework Template
% Last modified 8/20/2019 by Jeremy Buhler

\documentclass[11pt]{article}
\usepackage[left=0.7in,right=0.7in,top=1in,bottom=0.7in]{geometry}
\usepackage{fancyhdr} % for header
\usepackage{graphicx} % for figures
\usepackage{amsmath}  % for extended math markup
\usepackage{amssymb}
\usepackage[noend]{algpseudocode} % for pseudocode
\usepackage[plain]{algorithm} % float environment for algorithms

%%%%%%%%%%%%%%%%%%%%%%%%%%%%%%%%%%%%%%%%%%%%%%%%%%%%%%%%%%%%%%%%%%%%%%
% STUDENT: modify the following fields to reflect your the current
% homework number.  Do NOT include a name or ID or other personally
% identifying info, as we will use anonymized grading in GradeScope.

\newcommand{\HomeworkNumber}{1}

%%%%%%%%%%%%%%%%%%%%%%%%%%%%%%%%%%%%%%%%%%%%%%%%%%%%%%%%%%%%%%%%%%%%%%%%
% You can pretty much leave the stuff up to the next line of %%'s alone.

% create header and footer for every page
\pagestyle{fancy}
\fancyhf{}
\rhead{\textbf{Hwk \HomeworkNumber{}}}
\cfoot{\thepage}

% preferred pseudocode style
\algrenewcommand{\algorithmicprocedure}{}
\algrenewcommand{\algorithmicthen}{}

% ``do { ... } while (cond)''
\algdef{SE}[DOWHILE]{Do}{doWhile}{\algorithmicdo}[1]{\algorithmicwhile\ #1}%

% ``for (x in y ... z)''
\newcommand{\ForRange}[3]{\For{#1 \textbf{in} #2 \ \ldots \ #3}}

% these are common math formatting commands that aren't defined by default
\newcommand{\union}{\cup}
\newcommand{\isect}{\cap}
\newcommand{\ceil}[1]{\ensuremath \left\lceil #1 \right\rceil}
\newcommand{\floor}[1]{\ensuremath \left\lfloor #1 \right\rfloor}

%%%%%%%%%%%%%%%%%%%%%%%%%%%%%%%%%%%%%%%%%%%%%%%%%%%%%%%%%%%%%%%%%%%%%%

\begin{document}

\section{}
For this question, we will assume $Probe$ is a function $Probe: Class \times \mathbb{N} \rightarrow \mathbb{R}$

We want to find $a$ such that $n$ students across $X$ and $Y$ have anxiety less than $a$.

We will constrain our queries to those which could produce a valid median. Since there must be $n$ students below the median, we will choose some $k \in [0,n]$, and use queries $Probe(X, k)$ and $Probe(Y, n-k)$. 
Suppose we take $a_1 = Probe(X, k), a_2 = Probe(Y, n-k)$. Suppose, without loss of generality, that $a_1 > a_2$. Then $a_1$ is only a valid median value if there is no student in $Y$ with an anxiety level between $a_2$ and $a_1$, as this would mean that there are more than $k + n-k = n$ students with anxiety level $a_1$. We can measure this by taking $Probe(Y, n-k+1)$, and comparing against $a_1$. Since $a_1$ is invalid due to encompassing too many students in $Y$, we know that the correct value for $k$ must be smaller than what we have found, as any larger value for $k$ will increase $a_1$. Therefore our approach for the algorithm will be to binary search on $k$, eliminating half of the problem space until a solution is found.


\begin{algorithmic}[H]
    \State FindMedian$(n)$
        \Return FindMedianRec$(n, 0, n)$

    \State FindMedianRec$(n, mink, maxk)$
        \State $k \gets \frac{mink, maxk}{2}$
        \State $a_1 \gets $ Probe$(X, k)$
        \State $a_2 \gets $ Probe$(Y, n-k)$
        \If{$a_1 = a_2$}
            \State \Return $a_1$
        \ElsIf{$a_1 > a_2$}
            \If{$k = 0$} \Comment{All $X$ above $Y$}
                \State \Return $a_2$
            \EndIf
            \State NextA2 $\gets$ Probe$(Y, n-k+1)$
            \If{NextA2 $\leq a_1$} \Comment{$a$ is too high, decrease $k$}
                \If {$mink = k - 1$} \Comment{edge case where $k$ must be minimized}
                    \State \Return FindMedianRec$(n, mink, mink)$
                \Else
                    \State \Return FindMedianRec$(n, mink, k)$
                \EndIf
            \Else \Comment{$a_1$ is a valid median}
                \State \Return $a_1$
            \EndIf 
        \Else 
            \If{$k = n$} \Comment{ All $Y$ above $X$}
                \State \Return $a_1$
            \EndIf 
            \State NextA1 $\gets$ Probe$(X, k+1)$
            \If{NextA1 $\leq a_2$} \Comment{$a$ is too high, increase $k$}
                \If{$maxk = k + 1$} \Comment {edge case where $k$ must be maximized}
                    \State \Return FindMedianRec$(n, maxk, maxk)$
                \Else 
                    \State \Return FindMedianRec$(n, k, maxk)$
                \EndIf
            \Else \Comment{$a_2$ is a valid median}
                \State \Return $a_2$
            \EndIf 
        \EndIf
\end{algorithmic}

For each recursion, this algorithm will probe two to four times, that is, constant, and decreases the possible solutions by half. Therefore, the algorithm recurses $\log n$ times, and does constant probes at each recursion, so the number of probes $P(n)$ is $P(n) \in \Theta(\log n)$.

This algorithm is correct for all problem sizes $n$ which are a power of 2. 

\textbf{Proof: } by induction on $n$ where $n$ is a power of 2. 

Base case: 
Suppose $n=1$. Then there is exactly one student in each class. Suppose the student in $X$ has anxiety level $x$ and the student in $Y$ has anxiety level $y$. Suppose $x > y$. The algorithm will pick $k = 0$. The algorithm will recognize that all of $X$ is greater than $Y$ (via the condition that $k=0$), and will return $y$, which lies between the two students, and thus is the median. 
Suppose $x < y$. The algorithm will first pick $k=0$. The algorithm will probe for the level of $0$ in $X$, and will get some value less than $x$, and will probe for the level of $1$ in $Y$ and will get $y$. The algorithm will then probe for the next student in in $X$, who has anxiety $x$ which is less than $Y$. Recognizing that there exists a student with anxiety between the two polls, the algorithm will decide to increase $k$. Since the maximum value of $k$ is one greater than $k$, the algorithm will encounter the special case which sets both $maxk$ and $mink$ to $1$. Then, similarly to before, the algorithm will find that all of $Y$ is greater than all of $X$ (via the condition that $k = n$, and will thus return Probe$(X, 1)$, which is the median.


Inductive case:
Given $X$ and $Y$ and a choice $k \in [0..n]$, $a_1 = Probe(X, k), a_2 = Probe(Y, n-k)$ \\ 
Suppose $a_1 > a_2$. If $k = 0$, then all $X$ are greater than all $Y$ and the algorithm will find that the median is $a_2$, which encompasses all $n$ students of $Y$. Otherwise, if $Probe(Y, n-k+1) > a_1$, there exists no student in $Y$ with anxiety greater than $a_2$ and less than $a_1$, and therefore $a_1$ encompasses exactly $k + n - k = n$ students, and is thus the median. Otherwise, there exists a student with anxiety between $a_1$ and $a_2$, and therefore $a_1$ is too high to be the median, as it encompasses more than $n$ students. The algorithm then will search within the lower half of possible values of $k$, reducing the problem size by half. By the inductive hypothesis, the algorithm is correct for half the problem size, so the algorithm will find the median for the full problem size. 
This argument is identical for $a_2 > a_1$. 

Since the algorithm finds a median for class size 1, and finds a median for class size $2n$ if it does for class size $n$, it finds a median for all class sizes which are multiples of 2.

\pagebreak
\section{}
\subsection*{a.}
For a given square of tubes bounded by nodes $(i, j)$ and $(i+n, j+n)$, we can determine if it contains a leak by summing the inputs: $I = \sum_{k=j}^{j+n} f(t^{\rightarrow}_{i-1,k}) + \sum_{k=i}^{i+n} f(t^{\uparrow}_{k,j-1})$, and comparing against the sum of the outputs: $O = \sum_{k=j}^{j+n} f(t^{\rightarrow}_{i+n,k}) + \sum_{k=i}^{i+n} f(t^{\uparrow}_{k, j+n})$. If $O < I$, there is a leak somewhere within this square.

Then, we can define an algorithm to find leaks as so:

With $(i, j)$ being the lower left corner of the square with side length $n$ , and with $n$ guaranteed to be a power of 2,
\begin{algorithmic}[H]
    \State OutputMinusInput(i, j, n)
        \State $I \gets 0$
        \State $O \gets 0$
        \For{$k$ in $i..i+n$}
            \State $I \gets I + f(t^{\uparrow}_{k, j-1})$
        \EndFor
        \For{$k$ in $j..j+n$}
            \State $I \gets I + f(t^{\uparrow}_{i-1, k})$
        \EndFor
        \For{$k$ in $i..i+n$}
            \State $O \gets O + f(t^{\uparrow}_{k, j+n})$
        \EndFor
        \For{$k$ in $j..j+n$}
            \State $O \gets O + f(t^{\uparrow}_{i+n, k})$
        \EndFor
        \State \Return $O - I$
    \State FindLeaks(i, j, n):
        \If{n = 1} \Comment{Base case: found the leak}
            \State \Return $(i, j)$
        \Else
            \State NewN $\gets \frac{n}{2}$
            \If{OutputMinusInput$(i+NewN, j+NewN, NewN)$} \Comment{Quadrant 1}
                \State \Return FindLeaks$(i + NewN, j + NewN, NewN)$
            \ElsIf{OutputMinusInput$(i+NewN, j, NewN)$} \Comment{Quadrant 2}
                \State \Return FindLeaks$(i + NewN, j, NewN)$
            \ElsIf{OutputMinusInput$(i, j, NewN)$} \Comment{Quadrant 3}
                \State \Return FindLeaks$(i, j, NewN)$
            \ElsIf{OutputMinusInput$(i, j+NewN, NewN)$} \Comment{Quadrant 4}
                \State \Return FindLeaks$(i, j + NewN, NewN)$
            \EndIf
        \EndIf
\end{algorithmic}

Since OutputMinusInput measures each element of the perimeter of the square, it takes exactly $4n$ measurements.


Since each recursion divides the problem size in half by either halving the width or the height off the rectangle, the time complexity of the algorithm is expressed as the recurrence $T(n) = T(\frac{n}{4}) + 4n$. We can then use the master theorem to find the overall complexity. The regularity condition holds if $4\frac{n}{4} \leq k4n$ for some $k < 1$. If we let $k = \frac{1}{4}$, this inequality simplifies to $n \leq n$, which holds for all $n$, so by the master theorem, the overall complexity is $T(n) \in \Theta(n)$.

This algorithm is correct, and will find the leak, for any problem size $n$ which is a power of 2.

\textbf{Proof: } By recursion on $n$, over the powers of 2. 

Base case: 
Suppose $n = 1$. The algorithm will recognize that this is the position of the leak and return the appropriate coordinates.

Inductive case:
Suppose $n$ is a power of 2. The algorithm will divide the total network into four squares. Since both the width and height are powers of two, this will exactly divide the network into quadrants, with each quadrant having a side length of $\frac{n}{2}$. The algorithm will identify which of these quadrants contains a leak, and will then recurse into it. By the inductive hypothesis, the algorithm will find a leak for a problem of size $\frac{n}{2}$.

Since the algorithm is correct for $n=1$, and is correct for $n$ if it is correct for $\frac{n}{2}$, the algorithm is correct for all powers of 2.


\subsection*{b.}

For this problem, we will use the same approach as before, with the following adjustments. First, check the entire network. If $O = I$, the network has no leaks and the algorithm is finished. Otherwise, run the algorithm from part A with the following modifications: First, if both halves are found to have a leak, pick one arbitrarily. Second, after finding a leak, mark its location and 'fix' it, by replacing it with a pipe whose ouput is equal to its input, and thus eliminating one leak from the problem. Repeat this extended algorithm until there are no leaks. Since each repetition takes $\Theta(n)$ time and eliminates one leak, the algorithm runs in $\Theta(nk)$ time.

(It would technically be more efficient to recurse down both halves immediately, but not asymptotically faster, and would be harder to analyze)

\begin{algorithmic}[H]
\State FindAllLeaks(n):
    \State Leaks $\gets []$
    \While{OutputMinusInput$(0,0,n) < 0$}
        \State (LeakX, LeakY) $\gets$ FindLeaks(0,0,n)
        \State $f(t^{\uparrow}_{LeakX, LeakY}) \gets f(t^{\uparrow}_{LeakX, LeakY-1})$
        \State $f(t^{\rightarrow}_{LeakX, LeakY}) \gets f(t^{\rightarrow}_{LeakX-1, LeakY})$
    \EndWhile
    
\end{algorithmic}

\textbf{Claim: } The algorithm is correct for any $n$ which is a power of $2$ and any $k \in \mathbb{Nat}$.
\textbf{Proof: } By recursion on $k$. 

Base case:
Suppose $k = 0$. Then, the algorithm will immediately find that there are no leaks by checking the input and output of the entire network.

Inductive case:
Suppose there are $k$ leaks. The algorithm will run the sub-algorithm defined in part a on the entire network. We have proven previously that the subalgorithm is correct and will find a leak for any $n$ which is a power of 2. The algorithm will then eliminate this leak, leaving a problem of size $k-1$. By the inductive hypothesis, the algorithm is correct, and finds all leaks, for size $k-1$. 

Since the algorithm is correct for size $0$, and is correct for size $k$ if it is correct for size $k-1$, it is correct for all $k \in \mathbb{N}$. 


\end{document}
