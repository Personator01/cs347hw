%
% CSE Electronic Homework Template
% Last modified 8/20/2019 by Jeremy Buhler

\documentclass[11pt]{article}
\usepackage[left=0.7in,right=0.7in,top=1in,bottom=0.7in]{geometry}
\usepackage{fancyhdr} % for header
\usepackage{graphicx} % for figures
\usepackage{amsmath}  % for extended math markup
\usepackage{amssymb}
\usepackage{parskip}
\usepackage{amsthm}
\usepackage{tikz}
\usetikzlibrary{graphs, graphdrawing}
\usegdlibrary{layered}
\usepackage[noend]{algpseudocode} % for pseudocode
\usepackage[plain]{algorithm} % float environment for algorithms

%%%%%%%%%%%%%%%%%%%%%%%%%%%%%%%%%%%%%%%%%%%%%%%%%%%%%%%%%%%%%%%%%%%%%%
% STUDENT: modify the following fields to reflect your the current
% homework number.  Do NOT include a name or ID or other personally
% identifying info, as we will use anonymized grading in GradeScope.

\newcommand{\HomeworkNumber}{6}

%%%%%%%%%%%%%%%%%%%%%%%%%%%%%%%%%%%%%%%%%%%%%%%%%%%%%%%%%%%%%%%%%%%%%%%%
% You can pretty much leave the stuff up to the next line of %%'s alone.

% create header and footer for every page
\pagestyle{fancy}
\fancyhf{}
\rhead{\textbf{Hwk \HomeworkNumber{}}}
\cfoot{\thepage}

% preferred pseudocode style
\algrenewcommand{\algorithmicprocedure}{}
\algrenewcommand{\algorithmicthen}{}

% ``do { ... } while (cond)''
\algdef{SE}[DOWHILE]{Do}{doWhile}{\algorithmicdo}[1]{\algorithmicwhile\ #1}%

% ``for (x in y ... z)''
\newcommand{\ForRange}[3]{\For{#1 \textbf{in} #2 \ \ldots \ #3}}

% these are common math formatting commands that aren't defined by default
\newcommand{\union}{\cup}
\newcommand{\isect}{\cap}
\newcommand{\ceil}[1]{\ensuremath \left\lceil #1 \right\rceil}
\newcommand{\floor}[1]{\ensuremath \left\lfloor #1 \right\rfloor}

\newtheorem{thm}{Theorem}
\newtheorem{claim}{Claim}

%%%%%%%%%%%%%%%%%%%%%%%%%%%%%%%%%%%%%%%%%%%%%%%%%%%%%%%%%%%%%%%%%%%%%%

\begin{document}

\section{}

The problem is in NP.
\begin{proof}
    Let $l$ be an instance of the problem $D$. A certificate $c$ would be a subset of $P$ of size at most $k$ such that every email in $\mathcal{E}$ contains at least one phrase in $c$. Since $c$ is a subset of $P$, it is upper bounded by the size of $P$, and is thus linear in the problem size.

    Consider the following verifier for certificates which are subsets of $P$ with size at most $k$: \begin{itemize}
        \item For each $E \in \mathcal{E}$:
            \begin{itemize}
                \item For each $w \in c$:
                    \begin{itemize}
                        \item Search for $w$ in $E$. If found, continue to the next $w \in c$. If not found, reject $c$.
                    \end{itemize}
            \end{itemize}
        \item As all $E \in \mathcal{E}$ contain a phrase from $c$, $c$ is a certificate which shows $l$ is true.
    \end{itemize}

    Let $e$ be an upper bound on the size of an email. Then searching through an email is $O(e)$. Iterating through all words in $c$ is $O(c) = O(|P|)$, and iterating through all emails is $O(|E|)$. Therefore the verifier is $O(e|P||E|)$, and is thus polynomial in the problem size (and the certificate size since that is included).

    Since for every true instance of $D$ there exists a certificate polynomial in the size of the problem and a verifier polynomial in the size of the problem, $D$ is in NP.
\end{proof}

    Given an instance of $3SAT$, we will construct a corresponding instance of $D$. 

    Let $l$ be an instance of $3SAT$. It consists of a series of $n$ 3-clauses connected consisting of $m$ variables $P_{SAT}$ by logical AND. 

    The corresponding instance $p$ of $D$ will $k = m$, a spam vocabulary $P$ of size $2m$ consisting of $P_{SAT}$ and its inverses (that is, $x$ and $\lnot x$ are each distinct members of the vocabulary), and $n + m$ emails split into two groups. The $n$ emails correspond to the clauses of the boolean expression: for each 3-clause $l_i \in l$, the corresponding email $E_i \in \mathcal{E}$ will contain exactly the spam phrases in $l_i$ (with a variable and its negation counting as separate phrases). The $m$ other emails represent the constraint that every variable must be given a truth value, and only one of $x$ and $\neg x$ can be true, while the other must be false. This set of emails will be defined as, for every variable $x$, the email will contain $x$ and $\lnot x$.   

    Since a mapping will need to read every clause exactly once, and read the set of variables exactly once, and will produce exactly one email for every clause and every variable, the mapping will be polynomial in the amount of clauses and variables, and thus the problem size.

    An assignment of variables will correspond to a subset $Q$ of $P$. For each variable $x \in P_{SAT}$, if $x$ is true then the corresponding phrase in $P$ will in $Q$. If $x$ is false then the corresponding phrase $\lnot x$ will be in $Q$. 
 \\
 \\ 
 \\ 

\begin{claim}
    Given a true instance of $3SAT$, the corresponding instance of $D$ will be true.
\end{claim}

\begin{proof}
    Suppose we have a certificate $c$ for a true instance $l$ of $3SAT$. The corresponding instance $j$ of $D$ will be true with a certificate derived from $c$. 

    $c$ is a valid assignment of true and false to the variables such that, for each clause, at least one of the variables (after possible negation) is true. The corresponding instance $j$ consists of $m$ emails corresponding to the 3-clauses, and $n$ emails containing each variable phrase and its negation phrase. 

    As established previously, $c$ will map to a subset $Q$ of $P$, and for each variable-negation phrase pair, exactly one of the phrases will be selected. Therefore, for all of the emails of form $x, \lnot x$, one of the phrases will be in $Q$. As such, all of the $m$ emails with two phrases will have a phrase from $Q$.

    Suppose we have a 3-clause in $l$ of form $(x \lor y \lor z)$ (and suppose that negation has been folded into the the variable such that $x$ and $\lnot x$ would be separate variables with the constraint $x $ xor $ \lnot x$ is true. This is equivalent to having explicit negative values, but simplifies our analysis.). Since $c$ is an assignment of truth values to variables such that the entire CNF expression is true, it must assign at least one of $x$, $y$, or $z$ to be true. Therefore, for the corresponding email in $j$ which has form $x, y, z$, at least one of $z, y,$ and $z$ will be in $Q$, as $x \in Q$ if $x$ is true. As such, all of the $n$ emails with three phrases will contain a phrase from $Q$. 

    Since all emails contain a phrase from $Q$, and the size of $Q = m \leq k$, $Q$ is a certificate verifying that $j$ is a true instance of $D$. \\ \\ \\ 
\end{proof}

\begin{claim}
   Given an instance of $3SAT$ with corresponding true instance of $D$, the instance of $3SAT$ is true. 
\end{claim}

\begin{proof}
    Now suppose, for instance $l$ of $3SAT$ with corresponding instance $j$ of $D$, we had a valid $Q$ for $D$. The assignment $c$ corresponding to $Q$ will result in a true expression for $l$. 

    We will first show that, for every variable $x$, $Q$ contains exactly one of $x$ or $\lnot x$. Suppose $Q$ contained neither $x$ nor $\lnot x$. As established previously, $j$ will contain one email of the form $x, \lnot x$. If $Q$ does not contain $x$ or $\lnot x$, then this email contains no phrases from $Q$. This contradicts our assumption that $Q$ is valid, so $Q$ must contain at least one of $x$ and $\lnot x$. Now suppose $Q$ contained both $x$ and $\lnot x$. Since $Q$ is valid, for all of the $m$ emails of the form $y, \lnot y$, $Q$ must contain at least one phrase from that email. Besides the email $x, \lnot x$, there are $m-1$ emails of this form corresponding to the other variables, so since the phrase pairs contained in each of these emails is distinct, $Q$ must contain at least $m - 1$ additional phrases. This would make the size of $Q$ $2 + m-1 = m+1$. This contradicts our assumption that $Q$ is valid, as any valid $Q$ must have $|Q| \leq k = m$. Therefore $Q$ must contain at most one of $x, \lnot x$. As such, for every variable $x$, $Q$ contains exactly one of $x$ and $\lnot x$. 

    Now we can create an assignment of variables $c$ from $Q$ by the inverse process from before: If $x \in Q$, $x$ is assigned true. If $\lnot x \in Q$, $x$ is assigned false. As shown previously, exactly one of these is possible, so each variable will be assigned a single truth value. 

    Now consider any 3-clause of form $(x \lor y \lor z)$. (once again considering negations to be folded into the variables). Since for the corresponding email $x, y, z$, at least one of $x, y, z$ is in $Q$, at least one of $x, y, z$ is assigned true. Therefore every 3-clause will have at least one variable assigned true, and thus the variable assignment $c$ satisfies the expression. 

\end{proof}

Since an instance of 3SAT converts to an instance of $D$ in polynomial time, and an assignment of variables in $3SAT$ is valid if and only if its corresponding selection of phrases in $D$ is valid, $3SAT \leq_p D$, so $D$ is NP-hard. 

\section{}

A valid path is one from the start to the end which does not include both bars in any pair of forbidden bars.

The corresponding decision problem $D$ is: Does there exist a path from the start to the end which does not include any pairs of forbidden bars and has length at least k?.

The problem is in NP.
\begin{proof}
    Let $l$ be an instance of $D$ with $n$ bars.

    A certificate $c$ for $l$ consists of a sequence of pairs of bars which represent edges in the graph of bars.. Since any graph of bars can have at most $n(n-1)$ edges (in the case of a complete graph), this collection of edges, and thus the certificate, is polynomial in size relative to the size of the problem.

    Consider the following verifier.

    Given an instance $l$ of $D$ and a certificate $c$:
    \begin{itemize}
        \item Allocate a list of visited nodes.
        \item Store the start node as the current node.
        \item For every edge in $c$:
        \begin{itemize}
            \item Check that the edge is a valid edge in the graph and reject otherwise.
            \item Check that the edge starts at the current node and reject otherwise.
            \item For every node $b_i$ in the list of visited nodes:
                \begin{itemize}
                    \item Check that the pair consisting of the current node and $b_i$ are not in the list of forbidden nodes and reject if they are.
                \end{itemize}
            \item Store the current node in the list of visited nodes.
            \item Store the node at the end of the current edge as the current node.
        \end{itemize}
        \item Check that the current node is the end node and reject otherwise.
        \item Check that, for any visited node $b_i$, the pair consisting of the end node and $b_i$ is not in the forbidden list.
    \end{itemize}

    The maximum size of the list of visited nodes will be $n$. The maximum size of the forbidden nodes is every possible combination of two nodes, which is the binomial coefficient $\binom{n}{2} = \frac{n!}{2!(n-2)!} = \frac{n(n-1)}{2} \leq n^2$. Therefore checking that the current node does not violate any forbidden pair constraints is $O(n^3)$. Checking that the edge is a valid edge in the graph can be done by searching the list of edges, which as established previously, has maximum size $n(n-1) \leq n^2$. Therefore this check is $O(n^2)$. All other procedures in the loop are constant time, and the loop runs $|c| \leq n^2$ times, so the loop over all edges in $c$ is $O(|c|(n^3 + n^2)) = O(n^5)$. Additionally, one extra check of node constraints is done outside of the loop, so the total verification is $O(|c|(n^3 + n^2) + n^3) = O(n^5)$, which is polynomial in the size of the certificate and the size of the problem.

    Since for any instance $l$ of $D$ there exists a certificate $c$ which is polynomial size in the size of $l$, and any certificate can be verified in polynomial time, $D$ is in $NP$.
\end{proof}

Consider the following mapping from an instance $l$ of $3SAT$ to an instance $j$ of $D$:

Let $l$ be an instance of $3SAT$. This consists of a set of $m$ variables $P$ and a sequence of $n$ clauses of the form $(x \lor y \lor z)$, where $x, y, z$ are variables in $P$, or their negations. For the corresponding instance of $D$, let the set of bars be of size $2P + m + 1$. Starting from the furthest west, there will be one initial bar. Then, for each clause, add three bars with edges coming from the initial bar, which are marked with their corresponding variable. Then add one intermediate bar with edges from each of the three bars to it. This intermediate bar will be the initial bar for the next clause. Repeat until all clauses are added to the graph., and designate the final intermediate bar as the endpoint. For example, for the 3SAT instance $(x \lor y \lor \lnot w) \land (\lnot z \lor \lnot x \lor y) \land (x \lor \lnot w \lor z)$, the resulting graph will be 
\begin{figure}[H]
    \centering
    \begin{tikzpicture} 
        \begin{scope}[rotate=90,transform shape]
        \graph [layered layout, level distance = 15mm, sibling distance = 10mm, nodes = {draw, circle}, edges=thick] {
            "$start$" -> {"$x_1$", "$y_1$", "$\lnot w_1$"} -> "$i_1$" -> {"$\lnot z_2$", "$\lnot x_2$", "$y_2$"} -> "$i_2$" -> {"$x_3$", "$w_3$", "$z_3$"} -> "$end$"
        };
        \end{scope}
    \end{tikzpicture}
\end{figure}

Then, for every pair of bars $b_1, b_2$, where the variable corresponding to $b_1$ is the negation of the variable corresponding to $b_2$, put $(b_1, b_2)$ in the list of forbidden pairs.

The $k$ value for this problem will be $1$, as we do not care about the number of bars visited, only that there is a valid path which does not violate the constraints.

The conversion of an expression to a graph will need to read each variable in the expression exactly once, and will add a constant number of nodes and edges for each clause. Therefore the size of the graph, and thus the number of steps taken to build it, will be linear in the size of the input. The upper bound on possible combinations of nodes which will be placed in the forbidden list is the binomial coefficient $\binom{3n}{2} = \frac{(3n)!}{2!(3n - 2)!} = \frac{(3n)(3n - 1)}{2}$, so the size of the forbidden list, and thus the number of steps taken to build it, is polynomial in the size of the input. Therefore an instance of $3SAT$ is converted to an instance of $D$ in polynomial time.

Now, we will show that there exists a solution for an instance $l$ $3SAT$ if and only if there exists a solution for its corresponding instance $j$ of $D$. 

\begin{claim}
    Suppose $l$ is true. Then corresponding $j$ is true.  
\end{claim}
\begin{proof}
    

    Let $n$ be the number of clauses in $l$.

    If $l$ is true, then there exists an assignment of variables such that, for each clause in $l$, at least one of the variables (after possible negation) is true. The corresponding instance $j$ of $D$ will consist of a graph with a start node, then $n$ sets of 3 parallel nodes linked by intermediate nodes as described above. 

    Construct a path through this graph by the following process: \begin{itemize}
        \item Begin at the start node, and let that be the current node.
        \item For every clause and corresponding set of nodes of form $(x, y, z)$:
        \begin{itemize}
            \item For exactly one of the true variables, include the edge from the current node to the node corresponding to the variable.
            \item Include the edge from the variable node to the next intermediate node.
            \item Set the next intermediate node to be the current node.
        \end{itemize}
    \end{itemize}

    This is a valid path. It begins at the start node. It is continuous throughout, as since $l$ is true, then there must be at least one true variable in each clause. It ends at the end node, as the end node is the last intermediate node. Since this is a valid assignment of truth values, for every variable $x$, exactly one of $x$ and $\lnot x$ is true. Therefore the path must not visit nodes corresponding to both $x$ and $\lnot x$, as the path only passes through nodes corresponding to true variables. Since the forbidden pair list consists entirely of nodes corresponding to inverse variables, the path does not include any forbidden pairs. 
\end{proof}

\begin{claim}
    Suppose the corresponding instance $j$ of $D$ is true. Then $l$ is true. 
\end{claim}
\begin{proof}
    Since $j$ is true, then there exists a path from the start to the end nodes which does not include both of any forbidden pair. We will convert this path into a valid truth assignment of the variables of $l$. 

    Trace the path from the start. For each visited bar, set the corresponding variable true, or false if the bar corresponds to the negation of the variable. For any unassigned variables, their value does not matter, assign them the value true.

    The resulting assignment will be valid and will result in a true expression. Since the path does not include any forbidden pairs of a variable and its negation, for every assigned variable $x$ exactly one of $x$ and $\lnot x$ will be assigned true. The assignment will set at least one variable in every clause to be true (after negation), since it passes through every set of nodes corresponding to a clause, and therefore the assignment will result in a true expression.
\end{proof}

Since an instance of $3SAT$ converts to an instance of $D$ in polynomial time, and an instance of $3SAT$ is true if and only if its corresponding instance of $D$ is true, $3SAT \leq_p D$, and since $3SAT$ is NP-hard, $D$ is NP-hard, and since $D \in NP$, $D$ is NP-complete. 

Since its canonical decision problem $D$ is NP-complete, the original problem is an NP-optimization problem.









\end{document}
